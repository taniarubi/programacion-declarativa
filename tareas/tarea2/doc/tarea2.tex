\documentclass[letterpaper,11pt]{article}
% Soporte para los acentos.
\usepackage[utf8x]{inputenc}
\usepackage[T1]{fontenc}
\usepackage{amsmath}  
\usepackage{amssymb}   
\usepackage{amsthm}  
% Idioma español.
\usepackage[spanish,mexico, es-tabla]{babel}
% Modificamos los márgenes del documento.
\usepackage[lmargin=2cm,rmargin=2cm,top=2cm,bottom=2cm]{geometry}

\title{Facultad de Ciencias, UNAM \\
       Programación Declarativa \\ 
       Tarea 2: The Imperative is Dark and Full of Terrors}
\author{Rubí Rojas Tania Michelle}
\date{24 de febrero de 2020}

\begin{document}
\maketitle

\begin{enumerate}
    % Ejercicio 1.
    \item Demuestra las siguientes propiedades
    \begin{itemize}
        % Ejercicio 1.1
        \item sum . map double $=$ double . sum
        \begin{proof} 
            Inducción estructural sobre $xs$. 
            \begin{itemize}
                \item Caso base. \\ 
                $xs = []$. Este caso se cumple ya que 
                \begin{align*}
                    (sum \; . \; map \; double) \; []
                    &= sum \; (map \; double \; []) \\
                    &= sum \; []
                    && \text{def. de map} \\ 
                    &= 0
                    && \text{def. de sum} \\ 
                    &= 2 * 0
                    && \text{aritmética} \\ 
                    &= double \; 0
                    && \text{def. de double} \\ 
                    &= double \; (sum \; [])
                    && \text{def. de sum} \\ 
                    &= (double \; . \; sum) \; []
                \end{align*}

                \item Hipótesis de inducción.
                \begin{center}
                    sum . map double $=$ double . sum
                \end{center}

                \item Paso inductivo.
                \begin{align*}
                    (sum \; . \; map \; double) \; (x:xs)
                    &= sum \; (map \; double \; (x:xs)) \\ 
                    &= sum \; (double \; x \; : \; map \; double \; xs)
                    && \text{def. de map} \\
                    &= double \; x + sum \; (map \; double \; xs)
                    && \text{def. de sum} \\ 
                    &= double \; x + double \; (sum \; xs)
                    && \text{hipótesis de inducción} \\ 
                    &= 2 * x + 2 * (sum \; xs)
                    && \text{def. de double} \\ 
                    &= 2 * (x + sum \; xs)
                    && \text{distributividad} \\ 
                    &= double \; (x + sum \; xs)
                    && \text{def. de double} \\ 
                    &= double \; (sum \; (x:xs))
                    && \text{def. de sum} \\ 
                    &= (double \; . \; sum) \; (x:xs)
                \end{align*}
            \end{itemize}
        \end{proof}

        \newpage
        % Ejercicio 1.2
        \item sum . map sum $=$ sum . concat
        \begin{proof}
            Inducción estructural sobre $xs$.
            \begin{itemize}
                \item Caso base. \\ 
                $xs = []$. Este caso se cumple ya que 
                \begin{align*}
                    (sum \; . \; map \; sum) \; [] 
                    &= sum \; (map \; sum \; []) \\
                    &= sum \; []
                    && \text{def. de map} \\
                    &= sum \; (concat \; [])
                    && \text{def. de concat} \\ 
                    &= (sum \; . \; concat) \; []
                \end{align*}

                \item Hipótesis de inducción.
                \begin{center}
                    sum . map sum $=$ sum . concat
                \end{center}

                \item Paso inductivo.
                \begin{align*}
                    (sum \; . \; map \; sum) \; (x:xs)
                    &= sum \; (map \; sum \; (x:xs)) \\
                    &= sum \; (sum \; x \; : \; map \; sum \; xs) \\
                    &= sum \; x + sum \; (map \; sum \; xs)
                    && \text{def. de sum} \\ 
                    &= sum \; x + sum \; (concat \; xs)
                    && \text{hipótesis de inducción} \\ 
                    &= sum \; (x \; ++ \; concat \; xs)
                    && \text{propiedad de sum} \\ 
                    &= sum \; (concat (x:xs))
                    && \text{def. de concat} \\ 
                    &= (sum \; . \; concat) \; (x:xs)
                \end{align*}
            \end{itemize}
        \end{proof}

        % Ejercicio 1.3
        \item sum . sort $=$ sum
        \begin{proof}
            Inducción fuerte sobre la longitud de la lista.
            \begin{itemize}
                \item Caso base. \\ 
                $length \; xs = 0$. Esto quiere decir que la lista es vacía,
                por lo que este caso se cumple pues 
                \begin{align*}
                    (sum \; . \; sort) \; [] 
                    &= sum \; (sort \; []) \\ 
                    &= sum \; []
                    && \text{def. de sort} 
                \end{align*}

                \item Hipótesis de inducción. \\
                Supongamos que se cumple para todas las listas de longitud 
                menor a (x:xs).

                \item Paso inductivo.
                Notemos que la conmutatatividad de la suma hace que esto funcione.
                Sea {sort = quickSort}, entonces 
                \begin{align*}
                    (sum \; . \; sort) \; (x:xs)
                    &= sum \; (sort \; (x:xs)) \\
                    &= sum \; (sort \; (menorPrivote) \; ++ \; [x] \; \\ 
                    &\; ++ \; sort \; (mayorIgualPivote))
                    && \text{def. de quickSort} \\ 
                    &= sum \; (sort \; (menorPrivote)) \; + \; sum \; [x] \\
                    &\; ++ \; sum(sort (mayorIgualPivote)) 
                    && \textit{prop. de sum} \\ 
                    &= sum \; (menorPrivote) \; + \; sum \; [x] \; + \\
                    &\; sum (mayorIgualPivote)
                    && \text{H.I.} \\ 
                    &= sum (menorPrivote \; ++ \; [x] \; ++ \; mayorIgualPivote)
                    && \text{prop. de sum} \\
                    &= sum (x \; : menorPrivote \; ++ \; mayorIgualPivote)
                    && \text{prop. de $++$} \\ 
                    &= sum (x:xs)
                    && \text{def. de $++$}
                \end{align*}
            \end{itemize}
        \end{proof}
    \end{itemize}

    \newpage
    % Ejercicio 2.
    \item Demuestra o da un contraejemplo de cada una de las siguientes 
    propiedades.
    \begin{itemize}
        % Ejercicio 2.1
        \item take n xs $++$ drop n xs $=$ xs
        \begin{proof}
            Inducción sobre $n$.
            \begin{itemize}
                \item Caso base. \\ 
                $n = 0$. Este caso se cumple ya que 
                \begin{align*}
                    take \; 0 \; xs \; ++ \; drop \; 0 \; xs
                    &= [] ++ \; xs
                    && \text{def. de take y drop} \\ 
                    &= xs
                    && \text{def. de $++$}
                \end{align*}

                \item Hipótesis de inducción.
                \begin{center}
                    take n xs $++$ drop n xs $=$ xs
                \end{center}

                \item Paso inductivo. \\ 
                Tenemos dos casos
                \begin{itemize}
                    \item[i)] $n = n + 1 \land xs = []$
                    \begin{align*}
                        take \; (n+1) \; [] \; ++ \; drop \; (n+1) \; []
                        &= [] ++ \; []
                        && \text{def. de take y drop} \\
                        &= []
                        && \text{def. de $++$} 
                    \end{align*}

                    \item[ii)] $n = n + 1 \land xs = (x:xs)$ 
                    \begin{align*}
                        take \; (n+1) \; (x:xs) \; ++ \; drop \; (n+1) \; (x:xs)
                        &= (x \; : \; take \; n \; xs) \; 
                           ++ \; (drop \; n\; xs) \\
                        &= x \; : \; ((take \; n \; xs) \; 
                        ++ \; (drop \; n \; xs)) \\ 
                        &= (x:xs)
                    \end{align*}
                \end{itemize}
            \end{itemize}    
        \end{proof}

        % Ejercicio 2.2
        \item take m . take n $=$ take (min m n)
        \begin{proof}
            Sin pérdida de generalidad, asumimos que $m, n \geq 0$. \\ 
            Inducción sobre $m$.
            \begin{itemize}
                \item Caso base. \\ 
                $m = 0$. Este caso se cumple ya que 
                \begin{align*}
                    (take \; 0 \; . \; take \; n) \; xs
                    &= take \; 0 \; (take \; n \; xs) \\
                    &= [] 
                    && \text{def. de take} \\
                    &= take \; 0 \; xs 
                    && \text{def. de take} \\ 
                    &= take \; (min \; 0 \; n) \; xs 
                    && \text{def. de min} \\ 
                    &= (take \; (min \; 0 \; n)) \; xs
                \end{align*}

                \item Hipótesis de inducción.
                \begin{center}
                    take m . take n $=$ take (min m n)
                \end{center}
                
                \item Paso inductivo. \\ 
                Tenemos dos casos 
                \begin{itemize}
                    \item[i)] $m = m+1 \land xs = []$
                    \begin{align*}
                        (take \; (m+1) \; . \; take \; n) \; []
                        &= take \; (m+1) \; (take \; n \; []) \\ 
                        &= take \; (m+1) \; []
                        && \text{def. de take} \\ 
                        &= []
                        && \text{def. de take} \\
                        &= take \; (min \; (m+1) \; n) \; []
                        && \text{def. de take} \\ 
                        &= (take \; (min \; (m+1) \; n)) \; []
                    \end{align*}

                    \item[ii)] $n = n+1 \land xs = (x:xs)$ \\ 
                    Recordemos que \textit{min (x y) = min ((x+1) (y+1))}, 
                    es decir, el mínimo seguirá siendo el mínimo si es que 
                    a los números $x, y$ se les suma una unidad.
                    \begin{align*}
                        (take \; (m+1) \; . \; take \; n) \; (x:xs)
                        &= take \; (m+1) \; (take \; n \; (x:xs)) \\
                        &= take \; (m+1) \; (x \; : \; take \; (n-1) \; xs)
                        && \text{def. de take} \\
                        &= x \; : \; take \; m \; (take \; (n-1) \; xs)
                        && \text{def. de take} \\
                        &= x \; : \; take \; (min \; m \; (n-1)) \; xs 
                        && \text{H. I.} \\
                        &= take \; (min \; (m+1) \; n) \; (x:xs)
                        && \text{def. de take} \\ 
                        &= (take \; (min \; (m+1) \; n)) \; (x:xs)
                    \end{align*}
                \end{itemize}
            \end{itemize}    
        \end{proof}

        % Ejercicio 2.3
        \item map f . take n $=$ take n . map f
        \begin{proof}
            Inducción sobre $n$. 
            \begin{itemize}
                \item Caso base. \\ 
                $n = 0$. Este caso se cumple ya que 
                \begin{align*}
                    (map \; . \; take \; 0) \; xs 
                    &= map \; f \; (take \; 0 \; xs) \\ 
                    &= map \; f \; []
                    && \text{def. de take} \\ 
                    &= []
                    && \text{def. de map} \\ 
                    &= take \; 0 \; (map \; f \; xs)
                    && \text{def. de take} \\ 
                    &= (take \; 0 \; . \; map \; f) \; xs 
                \end{align*}

                \item Hipótesis de inducción.
                \begin{center}
                    map f . take n $=$ take n . map f
                \end{center}

                \item Paso inductivo. Tenemos dos casos
                \begin{itemize}
                    \item[i)] $n = n+1 \land xs = []$
                    \begin{align*}
                        (map \; . \; take \; (n+1)) \; []
                        &= map \; f \; (take \; (n+1) \; []) \\
                        &= map \; f \; []
                        && \text{def. de take} \\ 
                        &= []
                        && \text{def. de map} \\ 
                        &= take \; (n+1) \; []
                        && \text{def. de take} \\ 
                        &= take \; (n+1) \; (map \; f \; [])
                        && \text{def. de map} \\ 
                        &= (take \; (n+1) \; . \; map \; f) \; []
                    \end{align*}

                    \item[ii)] $n = n + 1 \land xs = (x:xs)$ 
                    \begin{align*}
                        (map \; . \; take \; (n+1)) \; (x:xs) 
                        &= map \; f \; (take \; (n+1) \; (x:xs)) \\
                        &= map \; f \; (x \; : \; take \; n \; xs)
                        && \text{def. de take} \\ 
                        &= f \; x \; : \; map \; f \; (take \; n \; xs)
                        && \text{def. de map} \\ 
                        &= f \; x \; : \; take \; n \; (map \; f \; xs)
                        && \text{hipótesis de inducción} \\ 
                        &= take \; (n+1) \; (f \; x \; : \; map \; f \; xs)
                        && \text{def. de take} \\ 
                        &= take \; (n+1) \; (map \; f \; (x:xs))
                        && \text{def. de map} \\ 
                        &= (take \; (n+1) \; . \; map \; f) \; (x:xs)
                    \end{align*}
                \end{itemize}
            \end{itemize}
        \end{proof}

        % Ejercicio 2.4
        \item filter p . concat $=$ concat . map (filter p)
        \begin{proof}
            Inducción sobre $xs$.
            \begin{itemize}
                \item Caso base. \\ 
                $xs = []$. Este caso se cumple ya que 
                \begin{align*}
                    (filter \; p \; . \; concat) \; [] 
                    &= filter \; p \; (concat \; []) \\ 
                    &= filter \; p \; []
                    && \text{def. de concat} \\ 
                    &= []
                    && \text{def. de filter} \\ 
                    &= concat \; []
                    && \text{def. de concat} \\ 
                    &= concat \; (map \; (filter \; p) \; [])
                    && \text{def. de map} \\  
                    &= (concat \; . \; map \; (filter \; p)) \; []
                \end{align*}

                \item Hipótesis de inducción.
                \begin{center}
                    filter p . concat $=$ concat . map (filter p)
                \end{center}

                \item Paso inductivo.
                \begin{align*}
                    (filter \; p \; . \; concat) \; (x:xs)
                    &= filter \; p \; (concat \; (x:xs)) \\ 
                    &= filter \; p \; (x \; ++ \; concat \; xs)
                    && \text{def. de concat} \\
                    &= (filter \; p) \; x \; ++ \; filter \; p \; (concat \; xs) 
                    && \text{distributividad} \\  
                    &= (filter \; p) \; x \; ++ \; concat \; (map \; (filter \; p) \; xs) 
                    && \text{hipótesis de inducción} \\ 
                    &= concat \; ((filter \; p) \; x \; : \; map \; (filter \; p) \; xs)
                    && \text{def. de concat} \\
                    &= concat \; (map \; (filter \; p) \; (x:xs))
                    && \text{def. de map} \\
                    &= (concat \; . \; map \; (filter \; p)) \; (x:xs)
                \end{align*}
            \end{itemize}
        \end{proof}
    \end{itemize}

    % Ejercicio 3.
    \item Consideremos la siguiente afirmación
    \begin{center}
        map (f . g) xs $=$ map f $\$$ map g xs
    \end{center}

    \begin{itemize}
        % Ejercicio 3.a
        \item[(a)] ¿Se cumple para cualquier $xs$? Si es cierta bosqueja la 
        demostración, en caso contrario, ¿qué condiciones se deben pedir sobre
        $xs$ para que sea cierta?

        \textsc{Solución:} Sí, se cumple para cualquier lista $xs$.
        Para justificarlo, demostraremos la propiedad usando inducción 
        estructural.

        \begin{proof}
            Inducción estructural sobre $xs$.
            \begin{itemize}
                \item Caso base.
                $xs = []$. Este caso se cumple ya que 
                \begin{align*}
                    map \; (f \; . \; g) \; []
                    &= []
                    && \text{def. de map} \\ 
                    &= map \; f \; []
                    && \text{def. de map} \\ 
                    &= map \; f \; \$ \; map \; g \; []
                \end{align*}

                \item Hipótesis de inducción.
                \begin{center}
                    map (f . g) xs $=$ map f $\$$ map g xs
                \end{center}

                \item Paso inductivo.
                \begin{align*}
                    map \; (f \; . \; g) \; (x:xs)
                    &= (f \; . \; g) \; x \; : \; map \; (f \; . \; g) \; xs
                    && \text{def. de map} \\ 
                    &= (f \; . \; g) \; x \; : \; map \; f \; \$ \; map \; g \; xs
                    && \text{hipótesis de inducción} \\
                    &= f \; (g \; x) \; : \;   map \; f \; \$ \; map \; g \; xs
                    && \text{def. de composición} \\
                    &= map \; f \; \$ \; g \; x \; : \; map \; g \; xs 
                    && \text{def. de map} \\
                    &= map \; f \; \$ \; map \; g \; (x:xs)
                    && \text{def. de map}
                \end{align*}
            \end{itemize}
        \end{proof}

        % Ejercicio 3.b
        \item[(b)] Intuitivamente, ¿qué lado de la desigualdad resulta más 
        eficiente? ¿Esto es cierto incluso en lenguajes con evaluación perezosa? 
        Justifica ambas respuestas.

        \textsc{Solución:} La expresión \textbf{map f $\$$ map g xs} hace dos 
        recorridos a la lista $xs$ al momento de ejecutarse (un recorrido 
        por cada función \textbf{map}), mientras que \textbf{map (f . g) xs}
        hace un sólo recorrido de la lista $xs$, el cual es realizado por 
        nuestra única función \textbf{map}. Es decir, \textbf{map (f . g) xs}
        hace la \textit{optimización} de reemplazar dos recorridos de la lista 
        $xs$ por uno solo.

        Esto es cierto incluso para lenguajes con evaluación perezosa ya que 
        en ambos casos debemos evaluar todos los valores de la lista, pues 
        es necesario gracias a la aplicación de la función \textbf{map}. 
        Así que si podemos evitar recorrer la lista más de una vez, es 
        mejor para nosotros (y nuestra complejidad en espacio).
    \end{itemize}
\end{enumerate}
\end{document}
