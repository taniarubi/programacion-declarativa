\documentclass[letterpaper,11pt]{article}
% Soporte para los acentos.
\usepackage[utf8x]{inputenc}
\usepackage[T1]{fontenc}
\usepackage{amsmath}  
\usepackage{amssymb}   
\usepackage{amsthm}  
% Código
\usepackage{listings}
\usepackage{xcolor}
\usepackage{hyperref}
\hypersetup{
    colorlinks=true,
    linkcolor=black,
    filecolor=black,      
    urlcolor=blue,
}
\urlstyle{same}

% Idioma español.
\usepackage[spanish,mexico, es-tabla]{babel}
% Modificamos los márgenes del documento.
\usepackage[lmargin=2cm,rmargin=2cm,top=2cm,bottom=2cm]{geometry}

\definecolor{codegreen}{rgb}{0,0.6,0}
\definecolor{codegray}{rgb}{0.5,0.5,0.5}
\definecolor{codepurple}{rgb}{0.58,0,0.82}
\definecolor{backcolour}{rgb}{0.95,0.95,0.92}

\lstdefinestyle{mystyle}{
    backgroundcolor=\color{backcolour},   
    commentstyle=\color{codegreen},
    keywordstyle=\color{magenta},
    numberstyle=\tiny\color{codegray},
    stringstyle=\color{codepurple},
    basicstyle=\ttfamily\footnotesize,
    breakatwhitespace=false,         
    breaklines=true,                 
    captionpos=b,                    
    keepspaces=true,                 
    numbers=left,                    
    numbersep=5pt,                  
    showspaces=false,                
    showstringspaces=false,
    showtabs=false,                  
    tabsize=2
}

\lstset{style=mystyle}

\title{Facultad de Ciencias, UNAM \\
       Programación Declarativa \\ 
       Tarea 3: Bringing you into the fold}
\author{Rubí Rojas Tania Michelle}
\date{12 de marzo de 2020}

\begin{document}
\maketitle
\begin{enumerate}
    % Ejercicio 1.
    \item Demuestra que las siguientes propiedades de los operadores de plegado:
    \begin{itemize}
        % Ejercicio 1.1
        \item[a)] foldr f e . map g $=$ foldr (f . g) e
        \begin{proof}
            Inducción estructural sobre $xs$.
            \begin{itemize}
                \item Caso base. \\ 
                $xs = []$. Este caso se cumple ya que 
                \begin{align*}
                    (foldr \; f \; e \; . \; map \; g) \; []
                    &= foldr \; f \; e \; (map \; g \; []) \\ 
                    &= foldr \; f \; e \; []
                    && \text{def. de map} \\ 
                    &= e
                    && \text{def. de foldr} \\ 
                    &= foldr \; (f \; . \; g) \; e \; []
                \end{align*}

                \item Hipótesis de inducción.
                \begin{center}
                    foldr f e . map g $=$ foldr (f . g) e
                \end{center}
                
                \item Paso inductivo.
                \begin{align*}
                    (foldr \; f \; e \; . \; map \; g) \; (x:xs)
                    &= foldr \; f \; e \; (map \; g \; (x:xs)) \\
                    &= foldr \; f \; e \; (g \; x \; : \; map \; g \; xs)
                    && \text{def. de map} \\ 
                    &= f \; (g \; x) \; (foldr \; f \; e \; (map \; g \; xs))
                    && \text{def. de foldr} \\ 
                    &= f \; (g \; x) \; (foldr \; (f \; . \; g) \; e \; xs)
                    && \text{hipótesis de inducción} \\ 
                    &= (f \; . \; g) \; x \; (foldr \; (f \; . \; g) \; e \; xs)
                    && \text{def. de composición} \\ 
                    &= foldr \; (f \; . \; g) \; e \; (x:xs)
                    && \text{def. de foldr} 
                \end{align*}
            \end{itemize}
        \end{proof}

        % Ejercicio 1.2
        \item[b)] foldl f e xs $=$ foldr (flip f) e (reverse xs)
        \begin{proof}
            Inducción estructural sobre $xs$.
            \begin{itemize}
                \item Caso base. \\ 
                $xs = []$. Este caso se cumple ya que
                \begin{align*}
                    foldl \; f \; e \; []
                    &= e
                    && \text{def. de foldl} \\ 
                    &= foldr \; (flip \; f) \; e \; []
                    && \text{def. de foldr} \\ 
                    &= \; foldr \; (flip \; f) \; e \; (reverse \; [])
                    && \text{def. de reverse}
                \end{align*}

                \item Hipótesis de inducción.
                \begin{center}
                    foldl f e xs $=$ foldr (flip f) e (reverse xs)
                \end{center}
                
                \item Paso inductivo.
                \begin{align*}
                    foldr \; (flip \; f) \; e \; (reverse \; (x:xs))
                    &= foldr \; (flip \; f) \; e \; (reverse \; xs ++ [x]) \\ 
                    &= foldr \; (flip \; f) \; (foldr \; (flip \; f) \; e \; [x]) \; (reverse \; xs) \\ 
                    &= foldr \; (flip \; f) \; ((flip \; f) \; [x] \; (foldr \; (flip \; f) \; e \; [])) \; (reverse \; xs) \\ 
                    &= foldr \; (flip \; f) \; ((flip \; f) \; [x] \; e) \; (reverse \; xs) \\
                    &= foldr \; (flip \; f) \; (f \; e \; x) \; (reverse \; xs) \\ 
                    &= foldl \; f \; (f \; e \; x) \; xs \\ 
                    &= foldl \; f \; e \; (x:xs)
                \end{align*}

                donde la justificación de los pasos es la siguiente:
                \begin{enumerate}
                    \item Aplicamos la definición de reverse.
                    \item Aplicamos lo demostrado en el inciso c).
                    \item Aplicamos la definición de foldr en (foldr (flip f ) e [x]).
                    \item Aplicamos la definición de foldr en (foldr (flip f ) e [])).
                    \item Aplicamos la definición de flip.
                    \item Aplicamos la hipótesis de inducción.
                    \item Aplicamos la definición de foldl.
                \end{enumerate}
            \end{itemize}
        \end{proof}

        % Ejercicio 1.3
        \item[c)] foldr f e (xs $++$ ys) $=$ foldr f (foldr f e ys) xs
        \begin{proof}
            Inducción estructural sobre $xs$.
            \begin{itemize}
                \item Caso base. \\ 
                $xs = []$. Este caso se cumple ya que 
                \begin{align*}
                    foldr \; f \; e \; ([] \; ++ \; ys)
                    &= foldr \; f \; e \; ys
                    && \text{def. de $++$} \\ 
                    &= foldr \; f \; (foldr \; f \; e \; ys) \; []
                    && \text{def. de foldr}
                \end{align*}

                \item Hipótesis de inducción.
                \begin{center}
                    foldr f e (xs $++$ ys) $=$ foldr f (foldr f e ys) xs
                \end{center}

                \item Paso inductivo.
                \begin{align*}
                    foldr \; f \; e \; ((x:xs) \; ++ \; ys)
                    &= foldr \; f \; e \; (x:(xs++ys))
                    && \text{def. de $++$} \\ 
                    &= f \; x \; (foldr \; f \; e \; (xs++ys))
                    && \text{def. de foldr} \\ 
                    &= f \; x \; (foldr \; f \; (foldr \; f \; e \; ys) \; xs)
                    && \text{hipótesis de inducción} \\ 
                    &= foldr \; f \; (foldr \; f \; e \; ys) \; (x:xs)
                    && \text{def. de foldr}
                \end{align*}
            \end{itemize}
        \end{proof}
    \end{itemize}

    % Ejercicio 2.
    \item Considera el siguiente tipo de dato algebraico en Haskell para definir
    árboles binarios.
    \begin{lstlisting}[language=Haskell]
        data Tree a = Void | Node (Tree a) a (Tree a)
    \end{lstlisting}

    Y la función \textit{foldT} que define el operador de plegado para la 
    estructura Tree, definido como sigue:
    \begin{lstlisting}[language=Haskell]
        foldT :: (b -> a -> b -> b) -> b -> Tree a -> b
        foldT _ v Void = v
        foldT f v (Node t1 r t2) = f t1' r t2'
          where t1' = foldT f v t1 
                t2' = foldT f v t2
    \end{lstlisting}

    \begin{itemize}
        % Ejercicio 2.1
        \item[a)] Da en términos de una función \textit{h} el patrón encapsulado
        por el operador \textit{foldT}.
        
        \textsc{Solución:} Queremos resolver la ecuación \textit{foldT f v = 
        foldr h b}. Por la Propiedad Universal de Fold, esta ecuación es 
        equivalente a 
        
        \begin{lstlisting}[language=Haskell]
            foldT f v Void = b
            foldT f v (Node t1 r t2) = h (foldT f v t1) r (foldT f v t2)
        \end{lstlisting}
        
        De la primera ecuación tenemos que $b = v$ por la definición de 
        $foldT$. De la segunda ecuación, calculamos a $h$ como sigue:
        \begin{center}
            foldT f v (Node t1 r t2) = h (foldT f v t1) r (foldT f v t2) \\
            $\Leftrightarrow$ definición de $foldT$ \\
            f (foldT f v t1) r (foldT f v t2) = 
            h (foldT f v t1) r (foldT f v t2) \\
            $\Leftrightarrow$ generalizando \textit{(foldT f v t1) r 
            (foldT f v t2)} como $tree$ \\
            f tree = h tree \\
            $\Leftrightarrow$ por extensionalidad de funciones \\
            f = h
        \end{center}
        
        Por lo tanto, el patrón encapsulado por el operador $foldT$ es 
        \begin{lstlisting}[language=Haskell]
            foldT f v = foldr f v
        \end{lstlisting}
        
        % Ejercicio 2.2
        \item[b)] Enuncia y demuestra la propiedad Universal del operador 
        \textit{foldT}, basándote en la Propiedad Universal vista en clase
        sobre el operador \textit{foldr} de listas.
        
        \begin{proof}
            La Propiedad Universal del operador $foldT$ es
            \begin{verbatim}
            foldT f v Void = v                       ⇔ foldT f v = foldr f v
            foldT f v (Node t1 r t2) = f t1' r t2'
              where t1' = foldT f r t1
                    t2' = foldT f r t2            
            \end{verbatim}
            
            Para demostrar la siguiente igualdad
            \begin{center}
                foldT f v = foldr f v
            \end{center}
            
            Como la igualdad es una instancia de la consecuencia de la propiedad
            universal, basta con demostrar las precondiciones de ésta, es decir,
            \begin{lstlisting}
                foldT f v Void = v
                foldT f v (Node t1 r t2) = f (foldT f v t1) r (foldT f v t2) 
            \end{lstlisting}
            
            que pueden ser verificadas fácilmente
            \begin{itemize}
                \item $Tree = Void$. Se cumple ya que 
                \begin{align*}
                    foldT \; f \; v \; Void &= v
                    && \text{def. de foldT}
                \end{align*}
                
                \item $Tree = (Node \; t1 \; r \; t2)$. Se cumple ya que 
                \begin{align*}
                    foldT \; f \; v \; (Node \; t1 \; r \; t2)
                    &= f \; t1' \; r \; t2'
                    && \text{definición de foldT} \\
                    &= f \; (foldT \; f \; v \; t1) \; r \; 
                    (foldT \; f \; v \; t2)
                    && \text{definición de $t1'$ y $t2'$}
                \end{align*}
            \end{itemize}
            
            con lo que quedan demostradas las dos precondiciones. Aplicando 
            la propiedad universal queda demostrada la propiedad original.
            
        \end{proof}

    \end{itemize}

    % Ejercicio 3.
    \item Calcula una definición eficiente para \textit{scanr} partiendo de la
    siguiente:
    \begin{center}
        scanr f e $=$ map (foldr f e) . tails
    \end{center}
    
    \textsc{Solución:} Calculamos la definición de la siguiente manera
    \begin{itemize}
        \item $xs = []$. Tenemos que 
        \begin{align*}
            scanr \; f \; e \; [] 
            &= map \; (foldr \; f \; e) \; (tails \; []) \\
            &= map \; (foldr \; f \; e) \; [[]]
            && \text{def. de tails} \\ 
            &= map \; (foldr \; f \; e) ([]:[])
            && \text{def. de $:$} \\
            &= foldr \; f \; e \; [] : map \; (foldr \; f \; e) \; []
            && \text{def. de map} \\ 
            &= e : map \; (foldr \; f \; e) \; []
            && \text{def. de foldr} \\ 
            &= e : []
            && \text{def. de map} \\
            &= [e]
            && \text{def. de $:$}
        \end{align*}
        
        \item $xs = (y:ys)$. Tenemos que 
        \begin{align*}
            scanr \; f \; e \; (y:ys) 
            &= map \; (foldr \; f \; e) \; (tails \; (y:ys)) 
            && \text{especificación de scanr} \\
            &= map \; (foldr \; f \; e) \; ((y:ys) : tails \; ys) 
            && \text{def. de tails} \\ 
            &= foldr \; f \; e \; (y:ys):map \; (foldr \; f \; e) \;(tails \;ys)
            && \text{def. de map} \\ 
            &= f \; y \; (foldr \; f \; e \; ys):map \; (foldr \; f \; e \;
            (tails \; ys)) \;
            && \text{def. de foldr} \\
            &= f \; y \; (foldr \; f \; e \; ys): scanr \; f \; e \; ys
            && \text{especificación de scanr} \\
            &= f \; y \; (head \; (scanr \; f \; e \; ys)):scanr \; f \; e \;ys
            && \text{head (scanr f z xs) == foldr f z xs.} 
        \end{align*}
    \end{itemize}
    
    La última igualdad la obtuve de 
    \url{https://hackage.haskell.org/package/base-4.12.0.0/docs/GHC-List.html}
    
    Por lo tanto, la definición más eficiente de \textit{scanr} sería
    \begin{lstlisting}[language=Haskell]
        scanr f e [] = [e]
        scanr f e (x:xs) = f x (head ys) : ys
          where ys = scanr f e xs
    \end{lstlisting}
    
    La especificación original requiere $O(n^{2})$ aplicaciones de una lista de
    longitud $n$, mientras que esta nueva definición sólo usa $O(n)$
    aplicaciones para una lista de longitud $n$.

    % Ejercicio 4.
    \item Considera la siguiente definición de la función \textit{cp} que 
    calcula el producto cartesiano.
    \begin{lstlisting}[language=Haskell]
         cp :: [[a]] -> [[a]]
         cp = foldr f e
    \end{lstlisting}

    \begin{itemize}
        % Ejercicio 4.1
        \item[a)] En la definición anterior, ¿quiénes son \textit{f} y 
        \textit{e}?
        
        \textsc{Solución:} La función $cp$ calcula el producto cartesiano de 
        una lista de listas, por lo que $f$ sería de la forma 
        \begin{center}
            f xs xss = $[$(x:ys) | x $\leftarrow$ xs, ys $\leftarrow$ xss $]$
        \end{center}

        y el elemento $e$ es igual a $[[]]$.
        
        % Ejercicio 4.2
        \item[b)] Dada la siguiente ecuación
        \begin{center}
            length . cp = product . map length
        \end{center} 

        en donde \textit{length} calcula la longitud de una lista y
        \textit{product} regresa el resultado de la multiplicación de todos los
        elementos de una lista. Demuestra que la ecuación es cierta, para esto
        es necesario reescribir ambos lado de la ecuación como instancias de 
        \textit{foldr} y ver que son idénticas.
        
        \textsc{Solución:} Debemos reescribir ambos lados de la ecuación como
        instancias de \textit{foldr}. El lado izquierdo de la expresión lo
        queremos expresar como 
        \begin{lstlisting}[language=Haskell]
            length . cp = foldr h b
        \end{lstlisting}
        
        Por la definición de $cp$, esta expresión se puede reescribir como
        \begin{center}
            (length . foldr f $[[]]$ \textbf{where} f xs xss =
            $[$(x:ys) | x $\leftarrow$ xs, ys $\leftarrow$ xss $]$) =
            foldr h b 
        \end{center}
       
        Por el teorema de fusión basta que se cumplan las siguientes dos 
        condiciones (sabemos que \textit{length} es una función estricta):
        \begin{lstlisting}[language=Haskell]
            length [[]]                       = b 
            (length (f zs zss) where f xs xss = [(x:ys) | x <- xs, ys <- xss])
                                              = h zs (length zss) 
        \end{lstlisting}
        
        De la primera ecuación obtenemos que $b = 1$ por la definición de 
        \textit{length}. La segunda ecuación puede transformarse en otra 
        condición más restrictiva, de la cual obtenemos
        \begin{center}
            length (f zs zss) = h zs (length zss) \\
            $\Leftrightarrow$ $|S \times T| = |S| \times |T|$ \\
            length zs $*$ length zss = h zs (length zss) \\
            $\Leftrightarrow$ generalizando \textit{length zss} como $ws$ \\
            length zs $*$ ws = h zs ws \\
            $\Leftrightarrow$ por extensionalidad de funciones \\
            ($*$) (length zs) = h zs \\
            $\Leftrightarrow$ def. de composición de funciones \\
            (($*$) . length) zs = h zs \\
            $\Leftrightarrow$ por extensionalidad de funciones \\
            (($*$) . length) = h 
        \end{center}
        
        Por lo tanto, 
        \begin{lstlisting}[language=Haskell]
            length . cp = foldr ((*) . length) 1
        \end{lstlisting}

        Ahora bien, el lado derecho de la expresión lo queremos expresar como 
        \begin{lstlisting}[language=Haskell]
            product . map length = foldr h b
        \end{lstlisting}
        
        Primero, expresaremos a \textit{map length} como una instancia de 
        \textit{foldr}. Sabemos que la función \textit{map} puede escribirse
        como
        \begin{center}
            map f = ($\lambda$y ys $\rightarrow$ (f y) : ys) $[]$
        \end{center}
        
        de donde 
        \begin{align*}
            \lambda y \; ys \; \rightarrow \; (f \; y) \; : \; ys
            &= \lambda y \; ys \; \rightarrow \; (:) \; (f \; y) \; ys
            && \text{haciendo prefijo a $:$} \\
            &= \lambda y  \; \rightarrow \; (:) \; (f \; y) 
            && \text{eta reducción} \\
            &= \lambda y \; \rightarrow \; ((:) \; . \; f) \; y
            && \text{def. de composición} \\ 
            &= ((:) \; . \; f)
            && \text{eta reducción}
        \end{align*}
        
        Por lo que, la función \textit{map} puede reescribirse como 
        \begin{center}
            map f = foldr ((:) . f) $[]$
        \end{center}
        
        Así,
        \begin{center}
            map length = foldr ((:) . length) $[]$
        \end{center}
        
        Regresando al lado derecho original, hay que tener en cuenta que 
        \begin{center}
            product = foldr ($*$) 1
        \end{center}
        
        Sabiendo esto, la expresión a la que queremos llegar se vería de la 
        forma
        \begin{center}
            foldr ($*$) 1 . foldr ((:) . length) [] = foldr h b 
        \end{center}
        
        Por el teorema de fusión de foldr, basta que se cumplan las siguientes
        dos condiciones (sabemos que product es estricta):
        \begin{lstlisting}
            foldr (*) 1 []              = b
            foldr (*) 1 (length x : xs) = h x (foldr (*) 1 xs)
        \end{lstlisting}
        
        De la primera ecuación obtenemos que $b = 1$ por la definición de 
        \textit{foldr}. De la segunda obtenemos
        \begin{center}
            foldr ($*$) 1 (length x : xs) = h x (foldr ($*$) 1 xs) \\
            $\Leftrightarrow$  def. de foldr \\
            ($*$) (length x) (foldr ($*$) 1 xs) = h x (foldr ($*$) 1 xs) \\
            $\Leftrightarrow$ generalizando (foldr ($*$) 1 xs) como $ys$\\ 
            ($*$) (length x) ys = h x ys \\
            $\Leftrightarrow$ por extensionalidad de funciones \\
            ($*$) (length x) = h x \\
            $\Leftrightarrow$ def. de composición de funciones \\
            (($*$) . length) x = h x \\
            $\Leftrightarrow$ por extensionalidad de funciones \\
            (($*$) . length) = h
        \end{center}
        
        Por lo tanto,
        \begin{lstlisting}[language=Haskell]
            product . map length = foldr ((*) . length) 1
        \end{lstlisting}
        
        Finalmente, podemos concluir que la ecuación original es cierta ya que 
        ambos lados de la igualdad son idénticos.
        
    \end{itemize}  
\end{enumerate}
\end{document}
